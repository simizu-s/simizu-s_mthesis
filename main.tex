\documentclass[11pt]{ist-thesis}
% \pagestyle{bachelorthesis}  % 卒論
%\pagestyle{ebachelorthesis}  % Graduation Thesis
\pagestyle{masterthesis}  % 修論(目安;38ページ以上)
% \pagestyle{emasterthesis}  % Master Thesis
%\pagestyle{draft}  % 未完成ドラフト
%\pagestyle{edraft}  % Draft

\title{ASTに基づくコードクローンの再分類手法の提案\\タイトル2行目}
\author{清水 ささら}
\supervisor{肥後 芳樹 教授}
%\supervisor{Professor Yoshiki Higo}
\deadline{令和8年2月9日} % 正規の提出期日
%\deadline{February Xth, XXXX}

\begin{document}

\titlepage

% タイトル中に改行が含まれる場合は,改行を取り除いたタイトルを再定義
\title{論文タイトル}

\abstract{
ここに概要を書く
}

\keyword{
  \hspace{7mm} 論文のキーワードその1\\
  \hspace{7mm} 論文のキーワードその2\\
  \hspace{7mm} 論文のキーワードその3\\
}

% 目次を出力
\toc

% 本文開始
% \section{1章タイトル}
% 1章本文
% \subsection{1.1節タイトル}
% 1.1節本文
% \subsubsection{1.1.1項タイトル}
% 1.1.1項本文

\section{はじめに}

\section{背景}
\subsection{コードクローン}
% Type分類について
コードクローンとは,プログラムテキスト中の一致または類似するコード片である~\cite{Inoue2001}. 
コードクローンにバグは含まれているとバグの修正漏れを引き起こす原因になり,ソースコードの保守性が低下する要因の一つとなる,
また,互いにクローン関係であるソースコード片のペアを,クローンペアと呼ぶ.

一般にコードクローンは,構文的な類似度に基づき,以下の4種類に分類される.

\begin{description}
    \item[Type1] 改行,コメント,空白を除いて一致
    \item[Type2] Type1に加えて,リテラル,識別子,型の違いを除いて一致 
    \item[Type3] Type2に加えて,文の挿入,削除,変更を除いて一致
    \item[Type4] 構文は異なるが同じ機能を提供する
\end{description}


\subsection{BigClneBench}
% 既存の大規模なベンチマークとその問題点
BigClneBenchは,クローン検出性能評価で用いられる大規模なベンチマークである.
BigClneBenchは,ビッグデータプロジェクト間リポジトリであるIJDataset2.0からマイニングされた真陽性クローン,偽陽性クローンからなるベンチマークである.
検索ヒューリスティックを用いて,ターゲットとなる機能を実装する可能性のあるコードスニペットを自動で特定し,
判定者によって手動で機能の真陽性,偽陽性のタグ付が行われている.

Type3及びType4の境界が曖昧であるため,BigCloneBenchでは,行単位での構文的類似度に基づいて0以上1未満の範囲で,以下のように分類している.

\begin{description}
    \item[Strongly Type3] 0.7以上1.0未満
    \item[Moderately Type3] 0.5以上0.7未満
    \item[Weakly Type3] 0.0以上0.5未満
\end{description}

BigClneBenchは,古典的なクローン検出手法において,クローン検出性能評価のベンチマークとして,幅広く使用されている.
既存の研究において,Weakly Type3クローンはType4クローンとして用いられる.

\section{提案手法}
本研究では,コードクローンの分類の新たな定義として抽象構文木を用いた分類を行う.

\subsection{抽象構文木(AST)}
ASTとはソースコードの構文構造を木構造で表現したデータ構造である.

本研究ではASTに基づいて新たな分類基準を定義した.
定義は以下の通りである.

\begin{description}
    \item[Type1] ASTが完全一致
    \item[Type2] 葉を除いてASTが一致 
    \item[Type3] 単文を除いてASTが一致
    \item[Type4] 上記に分類されないもの
\end{description}

\section{評価実験}

\section{実験結果と考察}

\section{妥当性への脅威}

\section{おわりに}


% 参考文献を参照する\cite{KamiyaTSE2002}

\acknowledgement{
}

% 参考文献を出力
\bibliographystyle{jplain}
\bibliography{reference}

\end{document}
